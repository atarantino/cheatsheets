\documentclass[landscape,11pt]{article}
\usepackage[utf8]{inputenc}
\usepackage[margin=1cm]{geometry}
\usepackage{multicol}
\usepackage[dvipsnames]{xcolor}
\usepackage{helvet}
\renewcommand{\familydefault}{\sfdefault}
\usepackage{tabularx}
\usepackage{tcolorbox}
\usepackage{microtype}
\usepackage{fancyhdr}
\usepackage{hyperref}
\hypersetup{hidelinks}
\urlstyle{same}

\definecolor{footgray}{gray}{0.35}

\setlength{\parindent}{0pt}
\setlength{\columnsep}{1cm}

\pagestyle{fancy}
\fancyhf{}
\renewcommand{\headrulewidth}{0pt}
\renewcommand{\footrulewidth}{0.2pt}
\setlength{\footskip}{16pt}
\fancyfoot[R]{\footnotesize\color{footgray} \url{https://github.com/atarantino/cheetsheets}}

\begin{document}
\begin{center}
{\Huge \textbf{VIM MOTIONS + OPERATORS CHEAT SHEET}}\\[5mm]
\end{center}

\begin{multicols}{2}

\section*{\textcolor{OliveGreen}{1. MOTIONS — Moving Through Text}}
Motions define \textbf{where} an operation acts. Combine them with operators to create powerful editing commands.

\renewcommand{\arraystretch}{1.1}
\begin{tabularx}{\linewidth}{@{}>{\ttfamily}l>{\raggedright\arraybackslash}X>{\raggedright\arraybackslash}X@{}}
\normalfont\textbf{Motion} & \textbf{Action} & \textbf{Example}\\
\hline
h / l & Move left / right & \texttt{3l} — move 3 chars right\\
j / k & Down / up line & \texttt{5j} — move 5 lines down\\
w / b & Next / previous word & \texttt{2w} — jump 2 words forward\\
0 / \$ & Start / end of line & \texttt{\$} — end of current line\\
f\{x\} & Find char right & \texttt{f(} — jump to next "("\\
t\{x\} & Until char (right) & \texttt{t(} — stop before "("\\
F\{x\}/T\{x\} & Backward find/until & \texttt{F[}, \texttt{T(}\\
; / , & Repeat last f/F/t/T & \texttt{;} — next match\\
\% & Matching bracket/brace & \texttt{\%} — move to opposite pair\\
8j & Move 8 lines down &  \\
\end{tabularx}
\renewcommand{\arraystretch}{1}

\vspace{2mm}
\begin{tcolorbox}[colback=OliveGreen!8, colframe=OliveGreen, boxsep=2pt, arc=1pt, fontupper=\small\itshape]
Tip: Motions can take counts — use numbers to repeat them quickly.
\end{tcolorbox}

\section*{\textcolor{YellowOrange}{2. TEXT OBJECTS — Acting on Chunks}}
Text objects define \textbf{semantic areas} inside or around words, quotes, or blocks.

\renewcommand{\arraystretch}{1.1}
\begin{tabularx}{\linewidth}{@{}>{\ttfamily}l>{\raggedright\arraybackslash}X>{\raggedright\arraybackslash}X@{}}
\normalfont\textbf{Text Object} & \textbf{Action} & \textbf{Example}\\
\hline
iw / iW & Inside word / WORD & \texttt{viw} — select a word\\
ip / ap & Inside / around paragraph & \texttt{dap}, \texttt{dip}\\
i" & Inside quotes & \texttt{ci"} — change quoted text\\
i( / a( & Inside / around parentheses & \texttt{di(}, \texttt{da()}\\
i\{\} & Inside braces & \texttt{ci\{\}}\\
\end{tabularx}
\renewcommand{\arraystretch}{1}

\vspace{2mm}
\begin{tcolorbox}[colback=YellowOrange!8, colframe=YellowOrange, boxsep=2pt, arc=1pt, fontupper=\small\itshape]
"Inside" excludes delimiters — "Around" includes them.
\end{tcolorbox}

\section*{\textcolor{BrickRed}{3. OPERATORS — What You Do}}
Operators define \textbf{the action}. Combine them with motions or text objects.

\renewcommand{\arraystretch}{1.1}
\begin{tabularx}{\linewidth}{@{}>{\ttfamily}l>{\raggedright\arraybackslash}X>{\raggedright\arraybackslash}X@{}}
\normalfont\textbf{Operator} & \textbf{Purpose} & \textbf{Example}\\
\hline
d & Delete & \texttt{dj}\\
c & Change (delete + insert) & \texttt{ct(}\\
y & Yank (copy) & \texttt{yiw}\\
p & Paste & \texttt{p}\\
> / < & Indent / unindent & \texttt{>ip}\\
= & Auto-indent & \texttt{=\%}\\
. & Repeat last change & \texttt{.}\\
\end{tabularx}
\renewcommand{\arraystretch}{1}

\vspace{2mm}
\begin{tcolorbox}[colback=BrickRed!8, colframe=BrickRed, boxsep=2pt, arc=1pt, fontupper=\small]
\textbf{The core pattern:} \texttt{OPERATOR + MOTION}\\[1mm]
Examples: \texttt{d2w} (delete two words), \texttt{y\$} (copy to EOL), \texttt{c\}} (change to end of paragraph).
\end{tcolorbox}

\section*{\textcolor{RoyalBlue}{4. VISUAL + CLIPBOARD BASICS}}
\renewcommand{\arraystretch}{1.1}
\begin{tabularx}{\linewidth}{@{}>{\ttfamily}l>{\raggedright\arraybackslash}X>{\raggedright\arraybackslash}X@{}}
\normalfont\textbf{Command} & \textbf{What It Does} & \textbf{Example}\\
\hline
v & Visual (character) & \texttt{vaw} — select a word\\
V & Visual (line) & \texttt{V\%} — select full block\\
"+y & Yank to system clipboard & \texttt{"+yy} — copy line globally\\
p & Paste after cursor & \texttt{p}\\
<leader> & Custom prefix (e.g. \texttt{<Space>}) & \texttt{<Space>y} — custom yank\\
\end{tabularx}
\renewcommand{\arraystretch}{1}

\vspace{2mm}
\begin{tcolorbox}[colback=RoyalBlue!8, colframe=RoyalBlue, boxsep=2pt, arc=1pt, fontupper=\small\itshape]
Vim's "registers" keep multiple clipboards: unnamed, system (\texttt{+}), or numbered.
\end{tcolorbox}

\section*{\textcolor{Purple}{5. EXAMPLE COMBOS}}
\renewcommand{\arraystretch}{1.1}
\begin{tabularx}{\linewidth}{@{}>{\ttfamily}l>{\raggedright\arraybackslash}X@{}}
\normalfont\textbf{Command} & \textbf{Effect}\\
\hline
ci( & Change inside parentheses\\
yiw & Yank word under cursor\\
dap & Delete entire paragraph\\
8j   & Move eight lines down\\
V\%y & Select whole code block and copy\\
\end{tabularx}
\renewcommand{\arraystretch}{1}

\vspace{2mm}
\begin{tcolorbox}[colback=Purple!8, colframe=Purple, boxsep=2pt, arc=1pt, fontupper=\small\itshape]
Editing feels like combos — motions are your movement, operators your attacks!
\end{tcolorbox}

\section*{\textcolor{Gray}{6. COOL CONCEPTS \& TOOLS}}
\renewcommand{\arraystretch}{1.1}
\begin{tabularx}{\linewidth}{@{}l>{\raggedright\arraybackslash}X@{}}
\textbf{Feature} & \textbf{Purpose / Benefit}\\
\hline
Relative line numbers & Show line distance for fast jumps (\texttt{8j}, \texttt{11k})\\
Quickfix list (\texttt{:copen}) & Review all matches or LSP references\\
Macros (\texttt{q a … q}) & Record/replay edits (\texttt{@a})\\
Harpoon plugin & Quick "sticky file" switching \\
Leader key mappings & Custom shortcuts for frequent actions\\
Registers & Separate internal and system clipboards\\
\end{tabularx}
\renewcommand{\arraystretch}{1}

\vspace{2mm}
\begin{tcolorbox}[colback=Gray!15, colframe=Gray, boxsep=2pt, arc=1pt, fontupper=\small\itshape]
Tip: You can automate repetitive changes using macros or quickfix navigation.
\end{tcolorbox}

\section*{\textcolor{Black}{7. QUICK PATTERNS TO REMEMBER}}
\renewcommand{\arraystretch}{1.1}
\begin{tabularx}{\linewidth}{@{}>{\ttfamily}l>{\raggedright\arraybackslash}X@{}}
\normalfont\textbf{Pattern} & \textbf{Meaning}\\
\hline
OPERATOR + MOTION & Perform action over motion range\\
vi\{\} & Visual select inside braces\\
di" & Delete inside quotes\\
dap & Delete paragraph\\
yiw, yaw & Yank word (inside / around)\\
@a & Execute macro a\\
:copen & Open quickfix window\\
. & Repeat last edit\\
\end{tabularx}
\renewcommand{\arraystretch}{1}

\vfill
\begin{center}
{\small ©2025 Vim Tutorial Reference}
\end{center}

\end{multicols}
\end{document}